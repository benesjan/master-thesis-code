\documentclass[11pt]{article}

\usepackage[czech]{babel}
\usepackage[utf8]{inputenc}
\usepackage[a4paper,width=150mm,top=25mm,bottom=30mm]{geometry}
\usepackage{setspace, enumitem, csquotes, amsmath, amsfonts, amsthm, soul, float, hyperref, fancyhdr, listings}
\usepackage[obeyspaces]{url}
\usepackage{graphicx}

\pagestyle{fancy}
\fancyhf{}
\rhead{Jan Beneš, A18N0051P}
\lhead{Semestrální práce}
\rfoot{Strana \thepage}

\setstretch{1.5}

\lstset{
columns=fullflexible,
%keywordstyle=\color{keywords},
frame=lines,
language=Python,
basicstyle=\small,
commentstyle=\it\footnotesize\color{Gray}
}

\renewcommand{\lstlistingname}{Kód}

\begin{document}
    \section*{Cíl práce}
    Cílem této semestrální práce je naprogramování a zdokumentování procesu předzpracování dat~\ref{sec:preprocessing},
    výpočtu příznakových vektorů~\ref{sec:features} a vyhodnocení statistik~\ref{sec:statistics}.


    \section{Předzpracování dat}\label{sec:preprocessing}

    \subsection{Popis datasetu}\label{subsec:dataset}
    Dataset se skládá celkem z 15 videí a stejného počtu anotačních souborů ve formátu \textit{json}.
    Videa jsou záznamem večerních událostí České Televize.
    Anotační soubory obsahují slovník, kde klíčem je jméno.
    Pod tímto klíčem je uložen seznam detekcí.
    Každá detekce je opět slovníkem.
    Tento slovník obsahuje číslo a oblast snímku, ve kterým se nachází obličej daného člověka.

    \subsection{Popis implementace předzpracování}
    Výstupem předzpracování by měl být dataset ve formátu \path{dataset/name/image_128x128.jpg}.
    Toho docílím následujícím algoritmem:

    \begin{enumerate}
        \item Nejprve proiteruji přes všechna videa.
        \item Dále proiteruji před jednotlivá jména a detekce.
        \item Pomocí balíku \textit{mtcnn-pytorch} detekuji všechny obličeje ve snímku příslušícím k detekci z minulého
        bodu a zároveň k detekcím obdržím významné body obličeje ("landmarks").
        \item Z nalezených detekcí vyberu tu, která má s původní detekcí největší hodnotu
        "Intersection over Union (IoU)" a zároveň tato hodnota musí být větší než práh \textbf{0.5}.
        \item Dále pomocí polohy významných bodů a předem definovaného kýženého rozložení významných bodů provedu
        takzvanou frontalizaci.
        Frontalizace využívá metody nejmenších čtverců k nalezení afinní transformace.
        Frontalizovaný obličej získáme aplikací této afinní transformace na původní snímek.
        \item Snímek uložím.
    \end{enumerate}


    Využil jsem detekci pomocí \textit{mtcnn-pytorch} místo původně dodaných detekcí z toho důvodu, že je žádané,
    aby byl obličej v jednotlivých snímcích vždy ve stejné pozici.
    Tato úprava zlepší funkčnost modelu.


    \section{Výpočet příznakových vektorů}\label{sec:features}
    Jakmile máme dataset v kýženém formátu, můžeme přistoupit k výpočtu příznakových vektorů.
    Tento výpočet probíhá následovně:
    \begin{enumerate}
        \item Proiteruji přes snímky v datasetu.
        \item Tento snímek a jeho převrácenou verzi uložím do pole dimenze 2x1x128x128.
        \item Takto vytvořené pole dále předložím již natrénovanému modelu ResNet-18.
        \item Na výstupu sítě získám příznakový vektor.
        \item Příznakový vektor a číslo odpovídající jménu uložím do pole.
        \item Výsledné pole všech příznakových vektorů uložím pomocí modulu \textit{h5py} na disk.
    \end{enumerate}

    \section{Vyhodnocení statistik}\label{sec:statistics}
    Druhým úkolem bylo naprahování vzdáleností pro každou dvojici příznakových vektorů a pro každý práh z množiny
    [0; 2; 0.05], kde 0.05 je krok.

    \subsection{Implementace}\label{subsec:implementaceprahovani}
    Z důvodu výpočetní náročnosti jsem úlohu paralelizoval.

    \begin{enumerate}
        \item Nejprve jsem vytvořil generátor, jehož návratovou hodnotou je dvojice intervalů.
        Tyto intervaly jsou výřezem z datasetu příznakových vektorů.
        Generátor postupně vrátí všechny možné dvojice intervalů.
        Tento přístup mi umožní projít přes všechny dvojice datasetu bez nutnosti načtení všech dvojic do operační
        paměti najednou.
        \item Dále pomocí metody \textit{multiprocessing.Pool.imap} paralelně předložím dvojice intervalů funkci.
        \item V této funkci vypočítám matici kosinových vzdáleností pro všechny dvojice a matici značek, která udává,
        zda je dvojice patří do stejné třídy ři nikoliv.
        \item Nyní proiteruji přes všechny prahy, naprahuji matici a porovnáním s maticí značek vypočtu hodnoty
        \textit{True positive}, \textit{True negative}, \textit{False positive} a \textit{False negative}.
        \item Hodnoty pro všechny dvojice intervalů sečtu a uložím na disk pomocí modulu \textit{h5py}.
    \end{enumerate}

    \begin{figure}[H]
        \centering
        \includegraphics{out/prft_fav-128_N1.eps}
        \caption{Průběh přesnosti, úplnosti a F-míry}
        \label{fig:prft}
    \end{figure}

    Na obrázku~\ref{fig:prft} vidíme průběh přesnosti, úplnosti a F-míry v závislosti na prahu.
    Jelikož přesnost je pro nulový práh jedna a zároveň hodnoty přesnosti s rostoucím prahem klesají, můžeme usoudit,
    že v datech ani v algoritmu není chyba.

    Optimální práh z hlediska F-míry je \textbf{0.62}.
    Pro tento práh je dosažena hodnota \textbf{0.96}, což je dobrý výsledek.

    \section{Závěr}
    V této práci jsem naprogramoval sadu skriptů, které nejprve zpracují data do požadovaného formátu, dále data
    předloží neuronové síti, na výstupech sítě vypočtou hodnoty potřebné pro vykreslení F-míry a na závěr
    F-míru vykreslí.

    Výsledné hodnoty odpovídají teorii, proto můžeme považovat výsledky za uspokojivé.

\end{document}
