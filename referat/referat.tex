\documentclass[11pt]{article}

\usepackage[czech]{babel}
\usepackage[utf8]{inputenc}
\usepackage[a4paper,width=150mm,top=25mm,bottom=30mm]{geometry}
\usepackage{setspace, enumitem, csquotes, amsmath, amsfonts, amsthm, soul, float, hyperref, fancyhdr, listings}
\usepackage{graphicx}

\pagestyle{fancy}
\fancyhf{}
\rhead{Jan Beneš, A18N0051P}
\lhead{Arcface}
\rfoot{Strana \thepage}

\setstretch{1.5}

\lstset{
columns=fullflexible,
%keywordstyle=\color{keywords},
frame=lines,
language=Python,
basicstyle=\small,
commentstyle=\it\footnotesize\color{Gray}
}

\renewcommand{\lstlistingname}{Kód}

\begin{document}
    \section{Výpočet příznakových vektorů}
    Prvním úkolem bylo vypočítání příznakových vektorů pro celý dataset.

    \subsection{Popis datasetu}
    Dataset se skládá celkem z 15 videí a stejného počtu anotačních souborů ve formátu \textit{json}.
    Videa jsou záznamem večerních událostí České Televize.
    Anotační soubory obsahují slovník, kde klíčem je jméno.
    Pod tímto klíčem je uložen seznam detekcí.
    Každá detekce je opět slovníkem.
    Tento slovník obsahuje číslo a oblast snímku, ve kterým se nachází obličej daného člověka.

    \subsection{Popis implementace}
    Samotný výpočet funguje následovně:
    \begin{enumerate}
        \item Nejprve proiteruji přes všechna videa.
        \item Dále proiteruji před jednotlivá jména a detekce.
        \item Ze snímku vyříznu oblast s obličejem, převedu oblast do greyscale a změnim velikost na 128x128.
        \item Tento snímek a jeho převrácenou verzi uložím do pole dimenze 2x1x128x128.
        \item Takto vytvořené pole dále předložím již natrénovanému modelu ResNet-18, čímž získám příznakový vektor.
        \item Příznakový vektor a číslo odpovídající jménu uložím do pole.
        \item Výsledné pole všech příznakových vektorů uložím pomocí modulu h5py na disk.
    \end{enumerate}

    \section{Naprahování vzdáleností}


\end{document}
