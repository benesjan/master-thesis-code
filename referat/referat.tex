\documentclass[11pt]{article}

\usepackage[czech]{babel}
\usepackage[utf8]{inputenc}
\usepackage[a4paper,width=150mm,top=25mm,bottom=30mm]{geometry}
\usepackage{setspace, enumitem, csquotes, amsmath, amsfonts, amsthm, soul, float, hyperref, fancyhdr, listings}
\usepackage{graphicx}

\pagestyle{fancy}
\fancyhf{}
\rhead{Jan Beneš, A18N0051P}
\lhead{Arcface}
\rfoot{Strana \thepage}

\setstretch{1.5}

\lstset{
columns=fullflexible,
%keywordstyle=\color{keywords},
frame=lines,
language=Python,
basicstyle=\small,
commentstyle=\it\footnotesize\color{Gray}
}

\renewcommand{\lstlistingname}{Kód}

\begin{document}
    \section{Výpočet příznakových vektorů}\label{sec:features}
    Prvním úkolem bylo vypočítání příznakových vektorů pro celý dataset.

    \subsection{Popis datasetu}\label{subsec:dataset}
    Dataset se skládá celkem z 15 videí a stejného počtu anotačních souborů ve formátu \textit{json}.
    Videa jsou záznamem večerních událostí České Televize.
    Anotační soubory obsahují slovník, kde klíčem je jméno.
    Pod tímto klíčem je uložen seznam detekcí.
    Každá detekce je opět slovníkem.
    Tento slovník obsahuje číslo a oblast snímku, ve kterým se nachází obličej daného člověka.

    \subsection{Implementace}\label{subsec:implementace}
    Samotný výpočet funguje následovně:
    \begin{enumerate}
        \item Nejprve proiteruji přes všechna videa.
        \item Dále proiteruji před jednotlivá jména a detekce.
        \item Ze snímku vyříznu oblast s obličejem, převedu oblast do greyscale a změnim velikost na 128x128.
        \item Tento snímek a jeho převrácenou verzi uložím do pole dimenze 2x1x128x128.
        \item Takto vytvořené pole dále předložím již natrénovanému modelu ResNet-18, čímž získám příznakový vektor.
        \item Příznakový vektor a číslo odpovídající jménu uložím do pole.
        \item Výsledné pole všech příznakových vektorů uložím pomocí modulu h5py na disk.
    \end{enumerate}

    \section{Naprahování vzdáleností}\label{sec:prahovani}
    Druhým úkolem bylo naprahování vzdáleností pro každou dvojici příznakových vektorů a pro každý práh z množiny
    [0; 2; 0.05], kde 0.05 je krok.

    \subsection{Implementace}\label{subsec:implementaceprahovani}
    Z důvodu výpočetní náročnosti jsem úlohu paralelizoval.

    \begin{enumerate}
        \item Nejprve jsem vytvořil generátor, jehož návratovou hodnotou je dvojice intervalů.
        Tyto intervaly jsou výřezem z datasetu příznakových vektorů.
        Generátor postupně vrátí všechny možné dvojice intervalů.
        Tento přístup mi umožní projít přes všechny dvojice datasetu bez nutnosti načtení všech dvojic do operační
        paměti najednou.
        \item Dále pomocí metody \textit{multiprocessing.Pool.imap} paralelně předložím dvojice intervalů funkci.
        \item V této funkci vypočítám matici kosinových vzdáleností pro všechny dvojice a matici značek, která udává,
        zda je dvojice patří do stejné třídy ři nikoliv.
        \item Nyní proiteruji přes všechny prahy, naprahuji matici a porovnáním s maticí značek vypočtu hodnoty
        \textit{True positive}, \textit{True negative}, \textit{False positive} a \textit{False negative}.
        \item Hodnoty pro všechny dvojice intervalů sečtu a uložím na disk pomocí modulu h5py.
    \end{enumerate}

    Níže vidíme grafy hodnot a F-míru.

    \begin{figure}[H]
        \centering
        \includegraphics{out/thresholds.eps}
        \caption{Průběh naakumulovaných hodnot}
        \label{fig:thresh}
    \end{figure}

    \newpage

    \begin{figure}[H]
        \centering
        \includegraphics{out/prft.eps}
        \caption{Průběh přesnosti, úplnosti a F-míry}
        \label{fig:prft}
    \end{figure}


\end{document}
